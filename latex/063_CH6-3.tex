\newpage
\section{控制系統設計與結果}
	最終的實體作品,我們選用Arduino UNO 作為開發板,VL53L0作為傳感器輸入,MG 996R伺服馬達進行輸出。


\subsection{控制系統}
Arduino板的控制程式如下:


\subsubsection{導入函式庫}
\begin{lstlisting}[language=C]
#include <Wire.h>       // 包含 I2C 通信的函式庫
#include <VL53L0X.h>    // 包含 VL53L0X 距離感測器的函式庫
#include <Servo.h>      // 包含控制伺服馬達的函式庫

VL53L0X sensor; // 宣告一個 VL53L0X 類別的物件,稱為 sensor
Servo motor;    // 宣告一個 Servo 類別的物件,稱為 motor
\end{lstlisting}


\subsubsection{設定PID 控制常數}
\begin{lstlisting}[language=C]
const float kp = 1.8;  // 比例增益
const float ki = 0.2;  // 積分增益
const float kd = 1;    // 微分增益
const float tt = 1000; // 時間延遲常數
\end{lstlisting}



\subsubsection{初始化誤差和積分項}
\begin{lstlisting}[language=C]
float error_sum = 0.0;         // 誤差積分項的初始值
float last_error = 0.0;        // 上一次的誤差
unsigned long last_time = 0;   // 上一次的時間
unsigned long last_control_time = 0; // 上一次控制的時間
\end{lstlisting}




\subsubsection{移動平均的參數設定}
\begin{lstlisting}[language=C]
const int numReadings = 10;    // 用於移動平均的讀數數量
int readings[numReadings];     // 存放 VL53L0X 的讀數
int readIndex = 0;             // 當前讀數的索引
int total = 0;         	    	// 總和
int average = 0;               // 平均值
\end{lstlisting}

			
			
		\subsubsection{導入函式庫}
			\begin{lstlisting}[language=C]
			\end{lstlisting}
			
			
			
			
		\subsubsection{導入函式庫}
			\begin{lstlisting}[language=C]
			\end{lstlisting}
			
			
			
		\subsubsection{導入函式庫}
			\begin{lstlisting}[language=C]
			\end{lstlisting}
			
			
			
		\subsubsection{導入函式庫}
			\begin{lstlisting}[language=C]
			\end{lstlisting}
			
			
		\subsubsection{導入函式庫}
			\begin{lstlisting}[language=C]
			\end{lstlisting}
			
			
			
		\subsubsection{導入函式庫}
			\begin{lstlisting}[language=C]
			\end{lstlisting}
		\subsubsection{導入函式庫}
			\begin{lstlisting}[language=C]
			\end{lstlisting}
		\subsubsection{導入函式庫}
			\begin{lstlisting}[language=C]
			\end{lstlisting}
			
		\subsubsection{導入函式庫}
			\begin{lstlisting}[language=C]
			\end{lstlisting}