\chapter{簡介}
\renewcommand{\baselinestretch}{10.0} %設定行距
\pagenumbering{arabic} %設定頁號阿拉伯數字

\fontsize{14pt}{2.5pt}\sectionef

本專題係藉由鋼求平衡台設計,探討其一,透過ODOO和GitHub進行協同設計管理、製造執行系統及其整合功能。其二,使用Geogebra、Solvespace與Onshape設計機構,導入CoppeliaSim並使用Python進行PID控制模擬,再使用自行維護之3D列印機製作鋼球平衡台之所需零件,以達成虛實整合之目標。其三,根據ODOO及Github之分析結果,探討協同作業之工作模式。

%-------------------研究流程------------------------------%
\section{研究流程}
材料分析軟體的應用在機械領域愈來越廣泛,能夠將繪製零件進行分析,但卻鮮少人知道分析是怎麼進行的,所以我們對四足機器人套用有限元素法,在其身上觀察有限元素法是如何計算出受力情況。\\


%-------------------研究環境------------------------------%
\section{研究環境}
材料分析軟體的應用在機械領域愈來越廣泛,能夠將繪製零件進行分析,但卻鮮少人知道分析是怎麼進行的,所以我們對四足機器人套用有限元素法,在其身上觀察有限元素法是如何計算出受力情況。\\
